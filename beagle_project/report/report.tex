\documentclass{article}

\usepackage{stmaryrd}
\usepackage{amsmath}
\usepackage{amssymb}
\usepackage{amsthm}
\usepackage{relsize} 
\usepackage{bm} 
\usepackage{IEEEtrantools}
\usepackage{graphicx}
\usepackage[font={small,it}, width=\textwidth]{caption}
\usepackage{subcaption}
\usepackage{hyperref}
\usepackage{cases}
\usepackage{xfrac}
\usepackage{comment}
\usepackage{framed}
\usepackage[ ddmmyyyy ]{datetime} 
\usepackage{fancyhdr}
\usepackage{enumitem}
\usepackage{cite}
\usepackage{float}
\usepackage{multirow}

\newcommand{\source}[1]{\caption*{\hfill Source: {#1}} }

\usepackage{mathtools}
\DeclareMathOperator{\sign}{sign}
\DeclareMathOperator{\sat}{sat}

\oddsidemargin = 20pt
\textwidth = 420pt

\hypersetup{
     colorlinks   = true,
     linkcolor    = blue
}

\title{The Beagle Project}
\author{Marcus Greiff}

\begin{document}
\maketitle

\section{Introduction}
This project is still in it's early stages, and the following document simply defines the equations of the DC motor model which is used to simulate the currently implemented regulators. A more detailed derivation and definitions will be added as soon as possible.
\section{Motor model}
\subsubsection*{Governing equations}
Assuming a simple DC motor model with constant magnetic field, Newton's and Kirchoff's laws can be used to derive the governing equations 
\begin{flalign}
\frac{d^2\theta(t)}{dt^2} &= \frac{1}{J}\Big(K_t i(t) - b\frac{d\theta (t)}{d t}\Big)\\
\frac{di(t)}{dt} &= \frac{1}{L}\Big(-Ri(t) + V(t) - K_e\frac{d\theta(t)}{dt}\Big)
\end{flalign}
\subsubsection*{Statespace form}
\begin{flalign}
\dot{\mathbf{x}}(t) &= \mathbf{A}\mathbf{x}(t) + \mathbf{B}\mathbf{u}(t)\\
\mathbf{y}(t) &= \mathbf{C}\mathbf{x}(t)
\end{flalign}
Let
\begin{equation}
\mathbf{x}(t) = \begin{bmatrix}\theta(t) & \dot{\theta}(t) & i(t) \end{bmatrix}^T, \qquad \mathbf{u}(t) = V(t)
\end{equation}
then
\begin{equation}
\mathbf{A} =\begin{bmatrix} 0 & 1 & 0 \\ 0 & -b/J & K_t/J  \\ 0 & -K_e/L & -R/L  \end{bmatrix}, \quad
\mathbf{B} =\begin{bmatrix} 0  \\ 0  \\ 1/L  \end{bmatrix}, \quad
\mathbf{C} = \mathbb{I}_{3\times 3}
\end{equation}
\subsubsection*{Laplace domain equivalent}
Let
\begin{equation}
\mathcal{L}\begin{Bmatrix}\Theta(t)\end{Bmatrix}_s=\Theta(s), \quad\mathcal{L}\begin{Bmatrix}i(t)\end{Bmatrix}_s=I(s), \quad\mathcal{L}\begin{Bmatrix}V(t)\end{Bmatrix}_s=V(s), \quad
\end{equation}
then
\begin{flalign}
s^2\Theta(s) &= \frac{1}{J}\Big(K_t I(s) - b s\Theta(s)\Big)\\
sI(s) &= \frac{1}{L}\Big(-R I(s) + V(s) - K_e s\Theta(s)\Big)
\end{flalign}
\subsubsection*{transfer function from $V(s)$ to $\Theta(s)$}
\begin{flalign}
G(s)_{U\rightarrow\Theta} &= \frac{\Theta(s)}{U(s)} =  \frac{K_t}{s((sL+R)(J s +b) +K_tK_e)}U(s)
\end{flalign}
Using a partial fraction decomposition, this may be written
\begin{flalign}
G(s)_{U\rightarrow\Theta} = \frac{A}{s} - \frac{B + Cs}{s^2 + D s + E} = \frac{A}{s} + \frac{B}{s^2 + D s + E}+\frac{Cs}{s^2 + D s + E}
\end{flalign}
\begin{flalign}
\frac{Kt}{(s*(R*b + Ke*Kt))} - \frac{(J*Kt*R + Kt*L*b)/(R*b + Ke*Kt) + (J*Kt*L*s)/(R*b + Ke*Kt)}{J*L*s^2 + (L*b + J*R)*s + R*b + Ke*Kt}
\end{flalign}
\subsubsection*{transfer function from $V(s)$ to $I(s)$}

\subsubsection*{LS-estimation}
Using a Bilinear z-transform, it is evident that
\begin{flalign}
\mathcal{Z}\begin{Bmatrix}G(s)\end{Bmatrix}_z = \frac{a_2z^2 + a_1z^1 + a_0}{z^3 + b_2z^2 + b_1z^1 + b_0}
\end{flalign}
Letting
\begin{equation}
\begin{cases}
\theta =  \begin{bmatrix}a_2 & a_1 & a_0 & b_2 & b_1 & b_0 \end{bmatrix}^T\\
\varphi(t-1) = \begin{bmatrix}u(t-1)& u(t-2) & u(t-3) & -y(t-1)& -y(t-2) & -y(t-3) \end{bmatrix}^T
\end{cases}
\end{equation}
we note that the discrete time difference equation can be written
\begin{equation}
y(t) = \varphi(t-1)^T\theta
\end{equation}

\subsection{Controllers}
\begin{flalign}
\dot {x}_1 &= x_2\\
\dot {x}_2 &= (-bx_2 + K_tx_3)/J  \\
\dot {x}_3 &= (-K_ex_2 -Rx_3+ u)/L
\end{flalign}
with
\begin{equation}
V(x_1, x_2, x_3) = \frac{1}{2}(x_1^2 + x_2^2 + x_3^2)
\end{equation}
\begin{flalign}
\frac{\partial V(x_1, x_2, x_3)}{\partial t} &= {x}_1\dot{x}_1 + {x}_2\dot{x}_2 + {x}_3\dot{x}_3\\
&= {x}_1x_2 + {x}_2(-bx_2 + K_tx_3)/J+ {x}_3(-K_ex_2 -Rx_3+ u)/L\\
&= {x}_1x_2 -bx_2^2/J + (K_t/J -K_e/L)x_2{x}_3 -Rx_3^2/L+ u/L
\end{flalign}
\newpage\bibliography{Bibliography}{}
\bibliographystyle{IEEEtran}
\end{document}